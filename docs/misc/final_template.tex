\documentclass{article}

\usepackage{amsfonts}
\usepackage{amsmath}
\usepackage{amsthm}
\usepackage{changepage}  % For adjustwidth environment
\usepackage{comment}
\usepackage{enumitem}
\usepackage{listings}
\usepackage{tikz}
\usepackage{todonotes}

\usetikzlibrary{automata,positioning}

% Syntax highlighting
\definecolor{codegreen}{rgb}{0,0.6,0}
\definecolor{codegray}{rgb}{0.5,0.5,0.5}
\definecolor{codepurple}{rgb}{0.58,0,0.82}
\definecolor{backcolour}{rgb}{0.95,0.95,0.92}
\lstdefinestyle{mystyle}{
    commentstyle=\color{codegreen},
    keywordstyle=\color{magenta},
    numberstyle=\footnotesize\color{codegray},
    stringstyle=\color{codepurple},
    basicstyle=\ttfamily\small,
    breakatwhitespace=false,
    breaklines=true,
    captionpos=b,
    keepspaces=true,
    numbers=left,
    numbersep=5pt,
    showspaces=false,
    showstringspaces=false,
    showtabs=false,
    tabsize=2,
    escapechar=@
}
\lstset{style=mystyle}

% Code style.
\newcommand{\C}[1]{\textup{\texttt{#1}}}

% Binary and unary all-uppercase text operators. \mathop gives too little space
% around the binary ones; e.g. \(P \op{AND} Q\) needs lots of space between
% those capital letters.
\newcommand\op[1]{{\ \mathrm{#1}\ }}
\newcommand\uop[1]{{\mathrm{#1}}}

\newcommand\FALSE{{\mathrm{F}}}
\newcommand\TRUE{{\mathrm{T}}}

\newcommand\AND{\wedge}
\newcommand\IFF{\iff}
\newcommand\IMPLIES{\implies}
\newcommand\NOT{\neg}
\newcommand\OR{\vee}

\newcommand{\emptylist}{[~]}
\newcommand\emptyString{\varepsilon}
\newcommand\Ints{{\mathbb{Z}}}
\DeclareMathOperator{\len}{len}
\newcommand\nats{{\mathbb{N}}}
\newcommand\Nats{\nats}
\newcommand{\N}{\mathbb{N}}
\newcommand\powerset{{\mathcal{P}}}

\newtheorem{lemma}{Lemma}

\newenvironment{indentone}{\begin{adjustwidth}{2em}{0em}}{\end{adjustwidth}}

\newcommand{\worthMarks}[1]{[#1 marks] \par}

\newcommand{\eqLang}{\operatorname{EQ}}
\newcommand{\timesChar}{\mbox{\( \times \)}}

\begin{document}

\begin{enumerate}
\item \worthMarks{10}

Suppose 
\( P : \nats \to \{ \TRUE, \FALSE \} \) 
is a predicate, and we
know that whenever \( P( n+1 ) \) is true, \( P( n ) \) is also true: in other
words, suppose
\[ \forall n \in \nats .\, (P( n+1 ) \allowbreak \IMPLIES P( n )) \] Prove that
for every natural number \( m \), if \( P( m ) \) is true, then \( P( k ) \) is
true for all natural numbers \( k < m \). For example, if \( P(2) \) is true,
then \( P(1) \) and \( P(0) \) must also be true.

Your proof must use induction or the fact that \( \le \) is a well-ordering on
\( \nats \).

\textit{Note: if you write an argument like the following, you will get 0
  marks, because it is not written as an induction or well-ordering proof (even
  though the idea behind it is correct): ``If \( P(m) \) is true,
  \( P( m-1 ) \) is also true (since \( P( m ) \IMPLIES P( m-1 ) \)), and
  similarly \( P( m-2 ) \) must be true since \( P( m-1 ) \IMPLIES P( m-2 ) \),
  and so on. In this way, we can see that
  \( P( m-1 ), P( m-2 ), \dotsc, P( 0 ) \) are all true.''}

\item \worthMarks{9}

Consider the following Python function:

\begin{lstlisting}[language=Python]
def f(n):
  if n < 3:
    return n
  else:
    return f(n-3)
\end{lstlisting}

\noindent Precondition: \( \C{n} \in \nats \)

\noindent Postcondition: Returns a natural number.

For \( n \in \nats \), let \( T( n ) \) be the total number of times \C{f} gets
called when \C{f(n)} is run (including the initial call). For example,
\( T(2) = 1 \) since \C{f(2)} never makes any recursive calls, and
\( T(6) = 3 \) since \C{f(6)} calls \C{f(3)} which calls \C{f(0)}.

\begin{enumerate}[label=(\alph*)]
\item \label{q:recurrence} Write a recurrence for \( T(n) \). Briefly
justify your recurrence in 1--2 sentences.

\item \label{q:closed} Write a closed form for \( T(n) \).

(You don't need to show how you found the closed form.)

\item Use complete induction to prove that your closed form in
part~\ref{q:closed} is the correct solution to your recurrence from
part~\ref{q:recurrence}.
\end{enumerate}

\item \worthMarks{20}

Given a list of 0's and 1's, $A$, we will say that $A$ is \textbf{balanced} if it contains exactly as many zeroes as ones. For example, $\emptylist$ and $[0, 1, 0, 1]$ are balanced, but $[0, 1, 0]$ is not. Consider the Python function \texttt{bal}, defined below, which decides whether a list is balanced.

\begin{lstlisting}[language=Python]
def bal(A):
    i = 0
    j = 1
    # Make a copy of A
    B = A.copy()
    while i < len(B) and j < len(B):
        if B[i] == B[j]:
            j += 1
        else:
            # swap the elements at indices j and i+1
            B[j], B[i+1] = B[i+1], B[j]
            i = i + 2
            j = i + 1
    return i == len(B)
\end{lstlisting}

\noindent Precondition: \C{A} is a list containing only 0's and 1's.

\noindent Postcondition: Returns True iff \C{A} is balanced.

\begin{enumerate}
    \item

    Use induction to prove that at the end of each iteration of bal's while loop, the sublist $B[:i]$ is balanced. (Recall that $B[:i]$ is a list containing all the elements of $B$ up to but not including index i. If $i \geq \len(B)$, then $B[:i] = B$)
    
    You may need to strengthen your induction by proving additional invariants.
    
    \item

    Prove that \texttt{bal} is partially correct. State any loop invariants you want to use at the start of your answer. You may assume those invariants without proof.

You should be confident that these invariants are actually true - if your proof relies on an invalid invariant, you may lose significant marks. (You should also avoid contrived invariants that trivially entail partial correctness. For example, `at the end of each iteration $k$, if $i_k \ge \len(B)$ or $j_k \ge \len(B)$, then $A$ is balanced if and only if $i$ is equal to $\len(B)$'. Or, even more to the point, `at the end of each iteration $k$, \texttt{bal} is partially correct'.)
\end{enumerate}

\item \worthMarks{8}

Let
\( \Sigma = \{ \C{0}, \C{1}, \C{2}, \C{3}, \C{4}, \C{5}, \C{6}, \C{7}, \C{8},
\C{9}, \timesChar, \C{=} \} \), so strings in \( \Sigma^* \) can have digits, as well as the symbols \( \timesChar \) and \( \C{=} \). Let \( \eqLang \subseteq \Sigma^* \) be the
set of true equations of the form \( a \timesChar b \C{=} c \), where \( a \),
\( b \) and \( c \) are strings of digits representing base-10 numbers and
\( \timesChar \) is multiplication. For example:
\begin{itemize}
\item \( \C{2\timesChar 3=6} \in \eqLang \)
\item \( \C{2\timesChar 03=00006} \in \eqLang \) (leading zeros are ignored)
\item \( \C{100\timesChar 37=3700} \in \eqLang \)
\item \( \C{2\timesChar 3=7} \not\in \eqLang \), since \( 2 \cdot 3 \not= 7 \).
\item \( \C{1\timesChar =1=1=\timesChar} \not\in \eqLang \), since it doesn't
have the form \( a \timesChar b \C{=} c \).
\end{itemize}
Prove that \( \eqLang \) is not regular.

\item \worthMarks{21}\label{q:r}

Let \( \Sigma = \{ \C{a}, \C{b} \} \) and let \( S \) be the smallest subset of
\( \Sigma^* \) satisfying:
\begin{itemize}
\item Base case: \( \emptyString \in S \).
\item First constructor case: If \( x \in S \) then \( x \C{aa} \in S \).
\item Second constructor case: If \( x, y \in S \) then \( x \C{b} y \in S \).
\end{itemize}
\begin{enumerate}[label=(\alph*)]
\item List two strings of length 4 that are in \( S \) and two strings of
length 4 that are in \( \{ \C{a}, \C{b} \}^* \) but not in \( S \). (You don't
need to explain your answer.)

\item \label{q:describe} Describe the set \( S \) in English. Don't
just translate the definition --- you should understand what strings are in
\( S \). In particular, your answer to this question should make it easy to
understand why your answers to parts \ref{q:drawDfsa} and
\ref{q:describeDfsaStates} are correct.

(For example, if the base case were \( \emptyString \in S \) and the
constructor cases were \( \C{a} x \in S \) and \( x \C{b} \in S \), a good
description would be: ``strings made of zero or more \C{a}'s followed by zero
or more \C{b}'s''.)

\item Write a regular expression denoting \( S \). You don't need to explain
your answer.

\item Prove that your description in part \ref{q:describe} is correct:
that every string in \( \Sigma^* \) is in \( S \) if and only if it matches
your description.

\item \label{q:drawDfsa} Draw a DFSA that accepts \( S \).

\item \label{q:describeDfsaStates} For each state \( q \) in your DFSA from
part~\ref{q:drawDfsa}, describe the set
\( L_q = \{ x \in \{ \C{a}, \C{b} \}^* \mid \delta^*( s, x ) = q \} \) where
\( s \) is your starting state.

(In other words, for each state, describe set of inputs that cause your DFSA to
finish in that state.)
\end{enumerate}
\end{enumerate}
\end{document}
